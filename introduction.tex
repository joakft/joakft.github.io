% Options for packages loaded elsewhere
% Options for packages loaded elsewhere
\PassOptionsToPackage{unicode}{hyperref}
\PassOptionsToPackage{hyphens}{url}
\PassOptionsToPackage{dvipsnames,svgnames,x11names}{xcolor}
%
\documentclass[
  letterpaper,
  DIV=11,
  numbers=noendperiod]{scrartcl}
\usepackage{xcolor}
\usepackage{amsmath,amssymb}
\setcounter{secnumdepth}{5}
\usepackage{iftex}
\ifPDFTeX
  \usepackage[T1]{fontenc}
  \usepackage[utf8]{inputenc}
  \usepackage{textcomp} % provide euro and other symbols
\else % if luatex or xetex
  \usepackage{unicode-math} % this also loads fontspec
  \defaultfontfeatures{Scale=MatchLowercase}
  \defaultfontfeatures[\rmfamily]{Ligatures=TeX,Scale=1}
\fi
\usepackage{lmodern}
\ifPDFTeX\else
  % xetex/luatex font selection
\fi
% Use upquote if available, for straight quotes in verbatim environments
\IfFileExists{upquote.sty}{\usepackage{upquote}}{}
\IfFileExists{microtype.sty}{% use microtype if available
  \usepackage[]{microtype}
  \UseMicrotypeSet[protrusion]{basicmath} % disable protrusion for tt fonts
}{}
\makeatletter
\@ifundefined{KOMAClassName}{% if non-KOMA class
  \IfFileExists{parskip.sty}{%
    \usepackage{parskip}
  }{% else
    \setlength{\parindent}{0pt}
    \setlength{\parskip}{6pt plus 2pt minus 1pt}}
}{% if KOMA class
  \KOMAoptions{parskip=half}}
\makeatother
% Make \paragraph and \subparagraph free-standing
\makeatletter
\ifx\paragraph\undefined\else
  \let\oldparagraph\paragraph
  \renewcommand{\paragraph}{
    \@ifstar
      \xxxParagraphStar
      \xxxParagraphNoStar
  }
  \newcommand{\xxxParagraphStar}[1]{\oldparagraph*{#1}\mbox{}}
  \newcommand{\xxxParagraphNoStar}[1]{\oldparagraph{#1}\mbox{}}
\fi
\ifx\subparagraph\undefined\else
  \let\oldsubparagraph\subparagraph
  \renewcommand{\subparagraph}{
    \@ifstar
      \xxxSubParagraphStar
      \xxxSubParagraphNoStar
  }
  \newcommand{\xxxSubParagraphStar}[1]{\oldsubparagraph*{#1}\mbox{}}
  \newcommand{\xxxSubParagraphNoStar}[1]{\oldsubparagraph{#1}\mbox{}}
\fi
\makeatother


\usepackage{longtable,booktabs,array}
\usepackage{calc} % for calculating minipage widths
% Correct order of tables after \paragraph or \subparagraph
\usepackage{etoolbox}
\makeatletter
\patchcmd\longtable{\par}{\if@noskipsec\mbox{}\fi\par}{}{}
\makeatother
% Allow footnotes in longtable head/foot
\IfFileExists{footnotehyper.sty}{\usepackage{footnotehyper}}{\usepackage{footnote}}
\makesavenoteenv{longtable}
\usepackage{graphicx}
\makeatletter
\newsavebox\pandoc@box
\newcommand*\pandocbounded[1]{% scales image to fit in text height/width
  \sbox\pandoc@box{#1}%
  \Gscale@div\@tempa{\textheight}{\dimexpr\ht\pandoc@box+\dp\pandoc@box\relax}%
  \Gscale@div\@tempb{\linewidth}{\wd\pandoc@box}%
  \ifdim\@tempb\p@<\@tempa\p@\let\@tempa\@tempb\fi% select the smaller of both
  \ifdim\@tempa\p@<\p@\scalebox{\@tempa}{\usebox\pandoc@box}%
  \else\usebox{\pandoc@box}%
  \fi%
}
% Set default figure placement to htbp
\def\fps@figure{htbp}
\makeatother





\setlength{\emergencystretch}{3em} % prevent overfull lines

\providecommand{\tightlist}{%
  \setlength{\itemsep}{0pt}\setlength{\parskip}{0pt}}



 


\KOMAoption{captions}{tableheading}
\makeatletter
\@ifpackageloaded{caption}{}{\usepackage{caption}}
\AtBeginDocument{%
\ifdefined\contentsname
  \renewcommand*\contentsname{Table of contents}
\else
  \newcommand\contentsname{Table of contents}
\fi
\ifdefined\listfigurename
  \renewcommand*\listfigurename{List of Figures}
\else
  \newcommand\listfigurename{List of Figures}
\fi
\ifdefined\listtablename
  \renewcommand*\listtablename{List of Tables}
\else
  \newcommand\listtablename{List of Tables}
\fi
\ifdefined\figurename
  \renewcommand*\figurename{Figure}
\else
  \newcommand\figurename{Figure}
\fi
\ifdefined\tablename
  \renewcommand*\tablename{Table}
\else
  \newcommand\tablename{Table}
\fi
}
\@ifpackageloaded{float}{}{\usepackage{float}}
\floatstyle{ruled}
\@ifundefined{c@chapter}{\newfloat{codelisting}{h}{lop}}{\newfloat{codelisting}{h}{lop}[chapter]}
\floatname{codelisting}{Listing}
\newcommand*\listoflistings{\listof{codelisting}{List of Listings}}
\makeatother
\makeatletter
\makeatother
\makeatletter
\@ifpackageloaded{caption}{}{\usepackage{caption}}
\@ifpackageloaded{subcaption}{}{\usepackage{subcaption}}
\makeatother
\usepackage{bookmark}
\IfFileExists{xurl.sty}{\usepackage{xurl}}{} % add URL line breaks if available
\urlstyle{same}
\hypersetup{
  pdftitle={Decisions as the Next Frontier of Software},
  pdfauthor={Joaquin Fernandez Tapia},
  colorlinks=true,
  linkcolor={blue},
  filecolor={Maroon},
  citecolor={Blue},
  urlcolor={Blue},
  pdfcreator={LaTeX via pandoc}}


\title{Decisions as the Next Frontier of Software}
\author{Joaquin Fernandez Tapia}
\date{2025-09-10}
\begin{document}
\maketitle

\renewcommand*\contentsname{Table of contents}
{
\hypersetup{linkcolor=}
\setcounter{tocdepth}{3}
\tableofcontents
}

\section{Introduction}\label{introduction}

Technology evolves by shifting the bottleneck. Each generation of tools
solves one constraint, only to reveal the next.

\begin{itemize}
\tightlist
\item
  In the era of \textbf{software engineering}, the challenge was making
  programs reliable and deployable at scale. Practices like automated
  testing, version control, and continuous integration turned fragile
  code into something industrial.\\
\item
  With the rise of \textbf{machine learning}, the challenge became
  managing data and models across their lifecycle. The response was
  pipelines and monitoring that allowed models to be deployed and
  updated reliably, not just prototyped in notebooks.
\end{itemize}

Now, with the surge of \textbf{AI and foundation models}, we face
another shift. The bottleneck is no longer writing software or even
building models --- those are increasingly commoditized. The bottleneck
is how to \textbf{manage decisions}: how to orchestrate models, rules,
and human input into reliable, adaptive, and auditable systems.

\section{Why Decisions Are the New Unit of
Management}\label{why-decisions-are-the-new-unit-of-management}

Every important business outcome comes down to decisions: approving a
transaction, setting a price, ranking a product, showing a
recommendation. These decisions are made not by a single model, but by
an interconnected system of data pipelines, predictive components,
business constraints, and feedback loops.

The question for the next decade is: how do we make these
\textbf{decision systems as manageable as software has become}?

That means:

\begin{itemize}
\tightlist
\item
  \textbf{Automation:} Decisions need pipelines, not manual
  interventions.\\
\item
  \textbf{Orchestration:} Multiple intelligent artifacts --- models,
  rules, heuristics --- must work together coherently.\\
\item
  \textbf{Adaptability:} With AI in the loop, decisions should evolve
  continuously as data and environments change.\\
\item
  \textbf{Trustworthiness:} Every decision should be explainable and
  auditable, not an opaque black box.\\
\item
  \textbf{Manageability:} Complexity must be tamed through modular
  design, observability, and reproducibility.
\end{itemize}

\section{What This Looks Like in
Practice}\label{what-this-looks-like-in-practice}

\begin{itemize}
\tightlist
\item
  \textbf{Built-in experimentation:} Any change to a decision process is
  tested, measured, and rolled back if necessary, the way code
  deployments are managed today.\\
\item
  \textbf{Continuous feedback:} Each decision feeds back into the
  system, enabling rapid adaptation rather than periodic retraining.\\
\item
  \textbf{Governance by design:} Logs, explanations, and audits are
  first-class outputs, not afterthoughts.\\
\item
  \textbf{Optimization under constraints:} Methods like bandits,
  reinforcement learning, or simulation are applied to balance
  exploration and exploitation, revenue and risk, short-term and
  long-term effects.\\
\item
  \textbf{Operational rigor:} Latency, cost-per-decision, and quality
  are tracked with the same discipline we apply to uptime and error
  rates in software systems.
\end{itemize}

\section{The Consistent Pattern}\label{the-consistent-pattern}

Across each wave, the value shifted:

\begin{itemize}
\tightlist
\item
  \textbf{Software:} The problem was deployment and reliability → the
  solution was DevOps.\\
\item
  \textbf{Models:} The problem was managing the data and model lifecycle
  → the solution was MLOps.\\
\item
  \textbf{Decisions:} The problem now is managing complexity,
  adaptability, and trust at the level of decisions → the solution is to
  treat decisions themselves as software artifacts, subject to
  automation, orchestration, and governance.
\end{itemize}

\section{Closing Thought}\label{closing-thought}

The AI wave makes this shift urgent. As models become cheap and
abundant, what will differentiate organizations is not who has the
biggest model, but who can \textbf{build decision systems that are fast,
adaptive, and trustworthy}.

In other words, the future is not just about better models. It's about
\textbf{making decisions themselves software-like}: automated,
orchestrated, continuously improving, and never spiraling out of
control.

That is where the next decade of value will come from.




\end{document}
